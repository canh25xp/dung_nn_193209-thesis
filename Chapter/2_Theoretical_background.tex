\documentclass[../Main.tex]{subfiles}
\begin{document}

\section{Front end}

\subsection{HTML}

HTML stands for Hypertext Markup Language, which is the most widely used language for writing web pages.

HTML is a markup language.
This means using HTML to mark up a text document with tags to instruct the web browser on how to structure it for display on the screen.

HTML can be thought of as helping us create the ``skeleton'' of a website.
Advantages of HTML:

\begin{itemize}
    \item It has abundant support resources within a vast user community
    \item It can smoothly operate on almost all modern web browsers.
    \item HTML markup tags are typically concise and highly consistent.
    \item It is open-source and completely free.
    \item HTML is a web standard governed by the W3C.
    \item It is easy to integrate with various backend languages (e.g. PHP, Node.js, Java, \dots).
\end{itemize}

\subsection{CSS}
CSS stands for Cascading Style Sheets, a language used to format elements created by markup languages like HTML.
In other words, while HTML structures elements on a website such as headings, tables, and paragraphs, CSS enriches and enhances these HTML elements by styling them, changing text colors, adding background colors, or modifying page layouts.

It can be understood that HTML creates the "skeleton" of a website, while CSS provides the colors, cosmetics, and appearance.

\subsection{JavaScript}

JavaScript is a widely used programming language for websites today, integrated and embedded into HTML to make websites more dynamic.
JavaScript acts as part of a webpage, executing client-side scripts that enable both user-side and server-side (Node.js) functionalities to create dynamic web pages.

JavaScript is an interpreted programming language with object-oriented capabilities.
It is one of the three main languages in web development and has interdependencies to build professional, dynamic websites.

The role of JavaScript is to manipulate HTML objects within the browser.
It can easily intervene in actions such as adding, deleting, or modifying CSS properties and HTML tags.
In other words, JavaScript is a client-side programming language for browsers.

JavaScript can fetch data from the server through APIs returned as JSON strings and display them on the user interface.
It can also retrieve data entered into forms on the interface and call APIs to save the data to the server.

\section{Backend}

\subsection{Java}

Java is one of the object-oriented programming languages used in software development, web applications, games, and mobile applications.
It was initiated by James Gosling and colleagues at Sun Microsystems in 1991, originally named Oak with the purpose of programming consumer electronic devices.

Unlike some other languages that run only in specific environments, Java programs can run effectively in various environments, known as "cross-platform" capability.
This hardware and operating system independence is demonstrated at both the source code and binary levels.

Java is a powerful and versatile programming language capable of developing websites, apps, AI, embedded systems, and more.

Java boasts several outstanding advantages:

\begin{itemize}
    \item It enforces strict data type requirements.
    \item Java tightly controls access to arrays and strings, preventing buffer overflow techniques.
          Thus, accesses do not exceed the size of arrays or strings.
    \item Memory allocation and deallocation processes are handled automatically.
    \item Error handling mechanisms simplify error recovery and management.
          However, Java also has some limitations compared to other languages:
    \item Like other interpreted languages, executing Java code is slower than compiled languages (though within acceptable limits).
\end{itemize}

\subsection{Spring framework}

Spring is a Java framework widely used by millions of developers for developing high-performance, easily testable, and reusable code applications.

Spring is an open-source framework developed, shared, and supported by a large community of users.
It is built upon two core design principles: Dependency Injection and Aspect-Oriented Programming.
The core features of Spring can be used to develop Java Desktop, mobile applications, and Java Web applications.
Spring's primary goal is to simplify the development of J2EE applications based on the POJO (Plain Old Java Object) model.

Spring is divided into several modules, each serving different application development purposes.

Benefits of the Spring framework include:

\begin{itemize}
    \item Spring allows developers to use POJOs, simplifying work without needing to deal with EJB, application threading, configuration, etc.
    \item Spring is modularly organized.
          Although it consists of many packages and classes, developers only need to focus on what they need, ignoring the rest.
    \item Spring supports various technologies such as ORM Framework, logging frameworks, JEE, scheduling libraries (Quartz and JDK timer), etc.
    \item The Spring Web module is designed based on the MVC pattern, providing comprehensive features that replace other web frameworks like Struts.
\end{itemize}

Spring provides projects like Spring Security, which enhances authentication and authorization mechanisms for projects, ensuring secure user authentication and access control.

Spring Boot: This framework facilitates rapid development and deployment of web applications.

\subsection{Hibernate}

Hibernate is an open-source Object Relational Mapping (ORM) library that helps Java developers write applications by mapping objects (POJOs) to relational database management systems, supporting object-oriented programming concepts with relational databases.

With Hibernate, developers do not need to work directly with the database as they would with JDBC.
Instead, they work with databases through classes that represent tables in the database.
Before Hibernate, developers had to use JDBC to write lengthy and verbose SQL queries, which often resulted in poor performance due to frequent opening and closing of connections.
With Hibernate, developers no longer need to write verbose SQL queries or manage connections manually multiple times.

\section{Database management system (DBSM)}

Database management system:

\begin{itemize}
    \item It is a system designed to automatically and systematically manage a specific volume of data.
          Management actions include editing, deleting, storing, and searching (retrieving information) within a defined dataset.
    \item It is a software package designed to define, manipulate, retrieve, and manage data within a database.
\end{itemize}

Roles of a Database Management System:

\begin{itemize}
    \item Provides an environment for creating databases: DBMS plays a role in providing users with a data definition language to describe, declare data types, and data structures.
    \item Provides methods for updating and exploiting data: DBMS provides users with a data manipulation language to express requests, update operations, and data exploitation.
          Data operations include Updating (inserting, editing, deleting data), Exploitation (searching, extracting data).
    \item Provides tools to control and manage access to the database to ensure the implementation of basic database requirements.
\end{itemize}

Thanks to its useful functions and high performance, many DBMSs have been developed with the aim of improving data processing capabilities for computer software, websites, etc.

Popular Database Management Systems today include MySQL, Oracle, SQLite, MongoDB, PostgreSQL \dots

For this problem, I choose to use MySQL as the database management system.

MySQL:

\begin{itemize}
    \item One of the most popular databases for web applications.
    \item This DBMS tool allows the selection of multiple storage engines.
    \item User-friendly interface and batch commands.
\end{itemize}

Advantages:

\begin{itemize}
    \item MySQL is free.
    \item Provides extensive functionality.
    \item Various user interfaces, including web and desktop app versions.
    \item It can operate on other databases like DB2 and Oracle.
\end{itemize}

Disadvantages:

\begin{itemize}
    \item Requires more time to work with MySQL compared to automated systems.
    \item Does not support XML or OLAP integration.
    \item Support is available in the free version but requires payment for certain features.
\end{itemize}

\section{Tools usec}

\subsection{IntelliJ IDE}
\begin{figure}[H]
    \centering
    \includegraphics[width=\textwidth]{Figure/ide.png}
    \caption{Tool IntelliJ IDE}
    \label{fig:ide}
\end{figure}
IntelliJ IDEA is an IDE used for Java programming (it also supports programming in some other languages like Node.js, Python \dots).

Generally, IntelliJ IDEA is similar to Eclipse because it is primarily used for Java but can also support other languages and has numerous supporting plugins.

IntelliJ IDEA has two editions: a free version (Community) and a paid version (Ultimate).
The paid version includes additional support for JavaScript, TypeScript, and plugins like GWT, Vaadin, and duplicate code detection.

\subsection{MySQL Workbench}

\begin{figure}[H]
    \centering
    \includegraphics[width=\textwidth]{Figure/mysql.png}
    \caption{Tool MySQL Workbench}
    \label{fig:mySQL}
\end{figure}

MySQL Workbench is a visual database design tool and integrated development environment (IDE) for MySQL.
It provides a graphical interface to design, model, and manage MySQL databases.

Similar to IntelliJ IDEA for Java development, MySQL Workbench is tailored specifically for MySQL databases but offers broader functionality.
It supports tasks such as database modeling, SQL development, and administration through a user-friendly graphical interface.

MySQL Workbench is available in two editions: Community (free) and Standard/Enterprise (paid), with the latter offering advanced features like database migration, performance monitoring, and collaboration tools.

\subsection{Postman}

\begin{figure}[H]
    \centering
    \includegraphics[width=\textwidth]{Figure/postman.png}
    \caption{Tool Postman}
    \label{fig:postman}
\end{figure}

Postman is currently one of the most popular tools used for API testing.
With Postman, you can call Rest APIs without writing any code.

Postman supports all HTTP methods (GET, POST, PUT, PATCH, DELETE, etc\dots)

Additionally, it allows you to save the history of requests, which is very convenient for reuse when needed.

Benefits of using Postman:

\begin{itemize}
    \item Using Collections: Postman allows users to create collections for their API calls.
          Each collection can have subfolders and multiple requests, which helps organize test sets.
    \item Collaboration: Collections and environments can be imported or exported, making it easy to share files.
    \item API Testing: Test the HTTP response status.
    \item Debugging: The Postman console helps check what data has been retrieved, making it easy to debug tests.
\end{itemize}

\section{Status survey}
\label{section:2.1}
Thông thường, khảo sát chi tiết về hiện trạng và yêu cầu của phần mềm sẽ được lấy từ ba nguồn chính, đó là (i) người dùng/khách hàng, (ii) các hệ thống đã có, (iii) và các ứng dụng tương tự.
Sinh viên cần tiến hành phân tích, so sánh, đánh giá chi tiết ưu nhược điểm của các sản phẩm/nghiên cứu hiện có.
Sinh viên có thể lập bảng so sánh nếu cần thiết.
Kết hợp với khảo sát người dùng/khách hàng (nếu có), sinh viên nêu và mô tả sơ lược các tính năng phần mềm quan trọng cần phát triển.

\section{Functional Overview}
\label{section:2.2}
Phần \ref{section:2.2} này có nhiệm vụ tóm tắt các chức năng của phần mềm.
Trong phần này, sinh viên lưu ý chỉ mô tả chức năng mức cao (tổng quan) mà không đặc tả chi tiết cho từng chức năng.
Đặc tả chi tiết được trình bày trong phần \ref{section:2.3}.

\subsection{General use case diagram}
\label{subsection:2.2.1}
Sinh viên vẽ biểu đồ use case tổng quan và giải thích các tác nhân tham gia là gì, nêu vai trò của từng tác nhân, và mô tả ngắn gọn các use case chính.

\subsection{Detailed use case diagram}
\label{subsection:2.2.2}
Với mỗi use case mức cao trong biểu đồ use case tổng quan, sinh viên tạo một mục riêng như mục \ref{subsection:2.2.2} và tiến hành phân rã use case đó.
Lưu ý tên use case cần phân rã trong biểu đồ use case tổng quan phải khớp với tên đề mục.

Trong mỗi mục như vậy, sinh viên vẽ và giải thích ngắn gọn các use case phân rã.

\subsection{Business process}
\label{subsection:2.2.3}
Nếu sản phẩm/hệ thống cần xây dựng có quy trình nghiệp vụ quan trọng/đáng chú ý, sinh viên cần mô tả và vẽ biểu đồ hoạt động minh họa quy trình nghiệp vụ đó.
Sinh viên lưu ý đây không phải là luồng sự kiện của từng use case, mà là luồng hoạt động kết hợp nhiều use case để thực hiện một nghiệp vụ nào đó.

Ví dụ, một hệ thống quản lý thư viện có quy trình nghiệp vụ mượn trả với mô tả sơ bộ như sau: Sinh viên làm thẻ mượn, sau đó sinh viên đăng ký mượn sách, thủ thư cho mượn, và cuối cùng sinh viên trả lại sách cho thư viện.
Một hệ thống có thể có một vài quy trình nghiệp vụ quan trọng như vậy.
\section{Functional description}
\label{section:2.3}
Sinh viên lựa chọn từ 4 đến 7 use case quan trọng nhất của đồ án để đặc tả chi tiết.
Mỗi đặc tả bao gồm ít nhất các thông tin sau: (i) Tên use case, (ii) Luồng sự kiện (chính và phát sinh), (iii) Tiền điều kiện, và (iv) Hậu điều kiện.
Sinh viên chỉ vẽ bổ sung biểu đồ hoạt động khi đặc tả use case phức tạp.
\subsection{Description of use case A}
\hfill
\subsection{Description of use case B}
\hfill

\section{Non-functional requirement}
\label{section:2.4}
Trong phần này, sinh viên đưa ra các yêu cầu khác nếu có, bao gồm các yêu cầu phi chức năng như hiệu năng, độ tin cậy, tính dễ dùng, tính dễ bảo trì, hoặc các yêu cầu về mặt kỹ thuật như về CSDL, công nghệ sử dụng, v.v.

%%%%%%%%%%%%%%%%%%%%%%%%%%%%%%%%%%%

\end{document}
